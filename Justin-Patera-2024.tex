%%%%%%%%%%%%%%%%%
% This is an sample CV template created using altacv.cls
% (v1.7, 9 August 2023) written by LianTze Lim (liantze@gmail.com). Compiles with pdfLaTeX, XeLaTeX and LuaLaTeX.
%
%% It may be distributed and/or modified under the
%% conditions of the LaTeX Project Public License, either version 1.3
%% of this license or (at your option) any later version.
%% The latest version of this license is in
%%    http://www.latex-project.org/lppl.txt
%% and version 1.3 or later is part of all distributions of LaTeX
%% version 2003/12/01 or later.
%%%%%%%%%%%%%%%%

%% Use the "normalphoto" option if you want a normal photo instead of cropped to a circle
% \documentclass[10pt,a4paper,normalphoto]{altacv}

\documentclass[10pt,letterpaper,ragged2e,withhyper]{altacv}
%% AltaCV uses the fontawesome5 and packages.
%% See http://texdoc.net/pkg/fontawesome5 for full list of symbols.

% Change the page layout if you need to
\geometry{left=1.25cm,right=1.25cm,top=1.5cm,bottom=1.5cm,columnsep=1.2cm}

% The paracol package lets you typeset columns of text in parallel
\usepackage{paracol}

% Change the font if you want to, depending on whether
% you're using pdflatex or xelatex/lualatex
% WHEN COMPILING WITH XELATEX PLEASE USE
% xelatex -shell-escape -output-driver="xdvipdfmx -z 0" sample.tex
\ifxetexorluatex
  % If using xelatex or lualatex:
  \setmainfont{Roboto Slab}
  \setsansfont{Lato}
  \renewcommand{\familydefault}{\sfdefault}
\else
  % If using pdflatex:
  \usepackage[rm]{roboto}
  \usepackage[defaultsans]{lato}
  % \usepackage{sourcesanspro}
  \renewcommand{\familydefault}{\sfdefault}
\fi

% Change the colours if you want to
\definecolor{SlateGrey}{HTML}{2E2E2E}
\definecolor{LightGrey}{HTML}{666666}
\definecolor{DarkPastelRed}{HTML}{450808}
\definecolor{PastelRed}{HTML}{8F0D0D}
\definecolor{GoldenEarth}{HTML}{E7D192}

\definecolor{MidPurple}{HTML}{7851A9}
\definecolor{LightPurple}{HTML}{8751A9}
\definecolor{DarkPurple}{HTML}{4B3B76}
\definecolor{LightBlue}{HTML}{516CA9}
\definecolor{SlateBlue}{HTML}{394C76}

\colorlet{name}{black}
\colorlet{tagline}{MidPurple}
\colorlet{heading}{DarkPurple}
\colorlet{headingrule}{LightPurple}
\colorlet{subheading}{MidPurple}
\colorlet{accent}{MidPurple}
\colorlet{emphasis}{SlateBlue}
\colorlet{body}{black}

% Change some fonts, if necessary
\renewcommand{\namefont}{\Huge\rmfamily\bfseries}
\renewcommand{\personalinfofont}{\footnotesize}
\renewcommand{\cvsectionfont}{\LARGE\rmfamily\bfseries}
\renewcommand{\cvsubsectionfont}{\large\bfseries}


% Change the bullets for itemize and rating marker
% for \cvskill if you want to
\renewcommand{\cvItemMarker}{{\small\textbullet}}
\renewcommand{\cvRatingMarker}{\faCircle}
% ...and the markers for the date/location for \cvevent
% \renewcommand{\cvDateMarker}{\faCalendar*[regular]}
% \renewcommand{\cvLocationMarker}{\faMapMarker*}


% If your CV/résumé is in a language other than English,
% then you probably want to change these so that when you
% copy-paste from the PDF or run pdftotext, the location
% and date marker icons for \cvevent will paste as correct
% translations. For example Spanish:
% \renewcommand{\locationname}{Ubicación}
% \renewcommand{\datename}{Fecha}


%% Use (and optionally edit if necessary) this .tex if you
%% want to use an author-year reference style like APA(6)
%% for your publication list
% % When using APA6 if you need more author names to be listed
% because you're e.g. the 12th author, add apamaxprtauth=12
\usepackage[backend=biber,style=apa6,sorting=ydnt]{biblatex}
\defbibheading{pubtype}{\cvsubsection{#1}}
\renewcommand{\bibsetup}{\vspace*{-\baselineskip}}
\AtEveryBibitem{%
  \makebox[\bibhang][l]{\itemmarker}%
  \iffieldundef{doi}{}{\clearfield{url}}%
}
\setlength{\bibitemsep}{0.25\baselineskip}
\setlength{\bibhang}{1.25em}


%% Use (and optionally edit if necessary) this .tex if you
%% want an originally numerical reference style like IEEE
%% for your publication list
\usepackage[backend=biber,style=ieee,sorting=ydnt,defernumbers=true]{biblatex}
%% For removing numbering entirely when using a numeric style
\setlength{\bibhang}{1.25em}
\DeclareFieldFormat{labelnumberwidth}{\makebox[\bibhang][l]{\itemmarker}}
\setlength{\biblabelsep}{0pt}
\defbibheading{pubtype}{\cvsubsection{#1}}
\renewcommand{\bibsetup}{\vspace*{-\baselineskip}}
\AtEveryBibitem{%
  \iffieldundef{doi}{}{\clearfield{url}}%
}


%% sample.bib contains your publications
\addbibresource{sample.bib}

\begin{document}
\name{Justin Patera}
\tagline{Computer Programmer \& Technician}
%% You can add multiple photos on the left or right
\photoR{2.8cm}{img/Justin-2024}
% \photoL{2.5cm}{Yacht_High,Suitcase_High}

\personalinfo{%
  % Not all of these are required!
  \email{serialhex@gmail.com}
  \phone{772-643-3621}
  %\mailaddress{Åddrésş, Street, 00000 Cóuntry}
  \location{Vero Beach, FL}
  %\homepage{www.homepage.com}
  %\twitter{@twitterhandle}
  \linkedin{justinpatera}
  \github{serialhex}
  %\orcid{0000-0000-0000-0000}
  %% You can add your own arbitrary detail with
  %% \printinfo{symbol}{detail}[optional hyperlink prefix]
  % \printinfo{\faPaw}{Hey ho!}[https://example.com/]

  %% Or you can declare your own field with
  %% \NewInfoFiled{fieldname}{symbol}[optional hyperlink prefix] and use it:
  % \NewInfoField{gitlab}{\faGitlab}[https://gitlab.com/]
  % \gitlab{your_id}
  %%
  %% For services and platforms like Mastodon where there isn't a
  %% straightforward relation between the user ID/nickname and the hyperlink,
  %% you can use \printinfo directly e.g.
  % \printinfo{\faMastodon}{@username@instace}[https://instance.url/@username]
  %% But if you absolutely want to create new dedicated info fields for
  %% such platforms, then use \NewInfoField* with a star:
  % \NewInfoField*{mastodon}{\faMastodon}
  %% then you can use \mastodon, with TWO arguments where the 2nd argument is
  %% the full hyperlink.
  % \mastodon{@username@instance}{https://instance.url/@username}
}

\makecvheader
%% Depending on your tastes, you may want to make fonts of itemize environments slightly smaller
% \AtBeginEnvironment{itemize}{\small}

%% Set the left/right column width ratio to 6:4.
\columnratio{0.6}

% Start a 2-column paracol. Both the left and right columns will automatically
% break across pages if things get too long.
\begin{paracol}{2}
\cvsection{Experience}

\cvevent{Senior Technician}{ACT Computers}{May 2022 -- Present}{Vero Beach, FL}
\begin{itemize}
\item Repair Computers
\item Schedule Service Calls
\item Generate Documentation for Employee Workflows
\end{itemize}

\cvevent{Computer Programmer}{Level 3 Inspection}{Feb 2016 - Feb 2022}{Stuart, FL}
Wrote and debugged code for various small-to-medium projects.
Did data automation tasks to reduce manual copy-paste from engineers.
Created processes to enable the processing of hundreds or thousands of parts,
including reporting to engineers and customers.

\cvevent{Programmer \& Assembly Technician}{Premier Citrus}{Dec 2018 - Feb 2019}{Vero Beach, FL}
\begin{itemize}
\item Prepped, wired and assembled proprietary computer control boxes for laser systems.
\item Did limited programming in C \& Assembly
\end{itemize}

\cvsection{Projects}

\cvevent{Haz Ion 3D Structured Light scanner}{Level 3 Inspection}{Jan 2017 - Dec 2017}{}
Wrote the operational code necessary for a 3D Structured Light scanner.
Mostly including data structures, timing routines, and numerous algorithms
to process hundreds of images in a few seconds.

\cvevent{Data Automation}{Level 3 Inspection}{Mar 2018 - Jun 2018}{}
Created automation to take data from Michigan,
and automatically upload, post-process, and output reports in CSV,
Excel and in Google Docs concurrently, both on a per-part and for
general part statistics.

\cvevent{Natural Light Labeling \& ApZ}{Premier Citrus}{Dec 2018 - Feb 2019}{}
Worked on controllers for the NLL (Natural Light Labeling) and ApZ laser technologies.
NLL is used to label fruit with a laser, reducing waste generated by stickers.
ApZ Technology helps fight citrus greening by using lasers to penetrate treatments into the tree’s vascular system


%% Switch to the right column. This will now automatically move to the second
%% page if the content is too long.
\switchcolumn

\cvsection{About Me}

\begin{quote}
Working hard to provide for my family.
Interested in technology.
More interested in maths.
Always looking for solutions to problems.
\end{quote}

\cvsection{Most Proud of}

\cvachievement{\faTrophy}{Husband \& Father of two}{Married over 10 years and eldest is in grade school}

\divider

\cvachievement{\faHeartbeat}{GitHub Hacktoberfest}{2017, 2018, 2019, 2020 and 2021.\linebreak Excercism.io Common Lisp path \& Apollo-11 code transcription}

\cvsection{Strengths}

\cvtag{Hard-working}
\cvtag{Eye for detail}\\
\cvtag{Adaptable}

\divider\smallskip

\cvsection{Serious About}

\cvtag{Formal Methods}
\cvtag{Functional Programming}
\cvtag{Testing}
\cvtag{Documentation}

\divider\smallskip

\cvsection{Programming\\
Languages}

\cvtag{Common Lisp}
\cvtag{C/C++}\\
\cvtag{Scheme}
\cvtag{Clojure}
\cvtag{\LaTeX}

\divider

\cvsection{Technologies}

\cvtag{\faLinux Linux}
\cvtag{\faWindows Windows}
\cvtag{\faGit Git}

\end{paracol}

%% So I'mma try this...
%% Have a little section (or a whole other page)
%% where I can put a cover letter...
\cvsection{Intro Letter}
\parindent = 20 pt

\noindent To Whom it may concern.

I found this listing, and it sounds very interesting.
The idea of designing robots to do menial tasks, like mow lawns, is awesome.
I know it's not a ground-breaking innovation - people have been using machines for millennium - but it is refreshing that technology is being used to make peoples lives easier, not harder.

While it may not be readily apparent, I do have some experience working outdoors.
For a few months between jobs I worked at Big Little Guy pressure washing in the hose management department (I ran the hose around, making sure it wasn't kinked and the job could get done effectively).
The owner's name is Richard Blunt and can be reached at (772) 480-6880 for a reference.

As a homeowner, I have been practically forced to acquaint myself with various power tools.
Doing so has enabled me to fix a number of issues around the house, including fixing my mailbox.
I also built a simple computer desk for myself.
There was also a time I assisted a friend building a custom projector mount that hung from the railing of a staircase.

And, of course, I do my own lawn work.

Working in software development I am familiar with needing detail and have worked with other engineers in solving problems.
When working at Level 3 Inspection I worked with mechanical engineers to ensure the data collected and processed was exactly what they needed.
Problem solving is something I enjoy, especially in a team setting where we can divide up the problems to solve them quicker, or troubleshoot together for particularly challenging problems.

Thank you for your consideration, and have an awesome day!

Justin Patera




\end{document}

